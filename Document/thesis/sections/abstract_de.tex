%% LaTeX2e class for student theses
%% sections/abstract_en.tex
%% 
%% Karlsruhe Institute of Technology
%% Institute for Program Structures and Data Organization
%% Chair for Software Design and Quality (SDQ)
%%
%% Dr.-Ing. Erik Burger
%% burger@kit.edu
%%
%% Version 1.3.3, 2018-04-17

\Abstract
Die Abschätzung der Korrelation von Attribute in einer Datenmenge ist einer der grundlegenden Aufgaben von Data Mining. Wenn man die Beziehung von Variablen kennt, dann kann man einige nützliche Ausgaben über zusätzliche und unbekannte Informationen folgern.\\
Normalerweise sind die Daten als Datenfluss verfügbar, d.h. Es ist unendlich und sogar evolutionär. Die zur-zeitigen schon erkennende Begriffe und Resultaten kann man in der Zukunft nicht mehr benutzen. Deshalb muss die Abschätzung der Korrelation ständig werden.\\
Ein anderes Problem ist die hohen Dimensionen. Die Daten enthalten oft mehr als 100 oder sogar 1000 Dimensionen, sodass es ist schwierig für das Rechnen der Daten. Es ist auch schwer für ein Mensch um Daten zu analysieren. Wenn es um die Korrelation über mehr als zwei Variablen geht, wächst das Rechnen der Datenmenge exponentiell an. Unsere Arbeit konzentriert sich zur Zeit nur auf Korrelation über zwei Variablen und das Problem auf die Korrelationen über mehr als zwei Variablen steht noch in der zukünftigen Arbeiten aus.\\
Das Ziel dieser Arbeit ist die Entwicklung von einer graphischen Schnittstelle für die Daten Wissenschaftler, um die Korrelation der Daten zu visualisieren. Mit dieser Schnittstelle, als zum Beispiel Web-Service, laden die Benutzer selbst Datenmenge hoch. Danach wird das Backend von System die Korrelationen von Attribute berechnen und einige Visualisierungen von Daten ausgeben. Es ist auch möglich für die Benutzer mit der Schnittstelle Parameter aufzustellen, um die Visualisierung zu verbessern. Zum Schluss bewerten wir die Vorteile und Nachteile dieser Schnittstelle und bestimmen die beste Visualisierung anhand verschiedene Situationen und Forderungen durch einige Anwendungsfälle.\\

